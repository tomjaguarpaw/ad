\documentclass[12pt]{article}
\usepackage{makecell,amsmath}
\usepackage[a4paper, total={6.5in, 8in}]{geometry}

\title{A note on efficient automatic differentiation of array programs
  in a linear language}

\begin{document}

\maketitle

In \cite{adml} the authors demonstrate how to differentiate an example
program, $y = f(x_1, x_2) = \ln(x_1)+x_1 x_2-\sin(x_2)$.

  \newcommand{\diff}[2]{
    \bar{v}_{#1} \frac{\partial v_{#1}}{\partial v_{#2}}
  }

\begin{tabular}[t]{ll}

  $v_1 = \ln{v_{-1}}$
  &
  \(\bar{v}_{-1}
  = \bar{v}_{-1} + \diff{1}{-1}
  = \bar{v}_{-1} + \bar{v}_1 / v_{-1}
  \) \\

  $v_2 = v_{-1} \times v_0$
  &
  \(\bar{v}_0
  = \bar{v}_0 + \diff{2}{0}
  = \bar{v}_0 + \bar{v}_2 \times v_{-1}
  \) \\

  &
  \(\bar{v}_{-1}
  = \diff{2}{-1}
  = \bar{v}_2 \times v_{0}
  \) \\

  $v_3 = \sin{v_0}$
  &
  \(\bar{v}_0
  = \diff{3}{0}
  = \bar{v}_3 \times \cos v_0
  \) \\

  $v_4 = v_1 + v_2$
  &
  \(\bar{v}_2
  = \diff{4}{2}
  = \bar{v}_4 \times 1
  \) \\

  &
  \(\bar{v}_1
  = \diff{4}{1}
  = \bar{v}_4 \times 1
  \) \\

  $v_5 = v_4 - v_3$
  &
  \(\bar{v}_3
  = \diff{5}{3}
  = \bar{v}_5 \times (-1)
  \) \\
  
  &
  \(\bar{v}_4
  = \diff{5}{4}
  = \bar{v}_5 \times 1
  \) \\
  
\end{tabular}




\begin{thebibliography}{9}

\bibitem{adml}
  Automatic differentiation in machine learning: a survey;
  Atilim Gunes Baydin, Barak A. Pearlmutter, Alexey Andreyevich Radul, Jeffrey Mark Siskind
  
https://arxiv.org/abs/1502.05767
  
\end{thebibliography}

\end{document}
