\documentclass[12pt]{article}
\usepackage{makecell,amsmath}
\usepackage[a4paper, total={6.5in, 8in}]{geometry}

\title{A note on efficient automatic differentiation of array programs
  in a pure language}

\author{Tom Ellis, Andrew Fitzgibbon, Simon Peyton Jones}

\newcommand{\dup}{\mathrm{dup}}

\begin{document}

\maketitle

\begin{abstract}
Reverse mode automatic differentiation (AD) is an algorithm which can
efficiently calculate the gradients of a numerical program.  When the
input program contains array indexing, standard AD algorithms will
produce a derivative program that involves array mutation.  In a
language that does not support mutation, a simple but inefficient
alternative is to use vectors which contain zero everywhere except at
a single index.

This note suggests an alternative approach by which reverse mode AD
can be implemented efficiently on numerical array programs in a pure
language.
\end{abstract}

\section{Introduction}

Presentations of reverse mode AD algorithms typically deal with the
accumulation of sensitivities by using mutation.  This poses a problem
for the designers of a language that is intended to be pure.  In a
pure language it should not be possible to mutate data in one part of
a program and have the mutation be observed elsewhere.  How can a
source-to-source reverse mode AD pass be added to such a language?  In
this note we describe an approach for resolving the apparent tension.

\section{Recapitulating reverse mode automatic differentiation}

In \cite{adml} the authors demonstrate how to perform source-to-source
reverse mode AD on program (\ref{adml-program}), written in a
first-order language

\begin{equation}
\label {adml-program}
y = f(x_1, x_2) = \ln(x_1)+x_1 x_2-\sin(x_2)
\end{equation}

Figure \ref{adml-diff} shows how the reverse mode derivative of
program (\ref{adml-program}) is generated.  Before differentiation the
program is prepared into a normal form in which there are no compound
expressions. Instead, every function application is made to a variable
(rather than a sub-expression).  This is achieved by the introduction
of fresh variables.  The normal form is similar to ANF from functional
programming and SSA from assembly languages.  The preparation step is
a form of book-keeping that allows the subsequent AD pass to be
simpler.  If this step is not performed explicitly then the AD pass
will have to perform it implicitly.

  \newcommand{\diff}[2]{
    \bar{v}_{#1} \frac{\partial v_{#1}}{\partial v_{#2}}
  }

\begin{figure}[t]
\center
\begin{tabular}[t]{ll|lll}

  $v_{-1}$ & $= x_1$
  &
  $\bar{x}_1$ & $= \bar{v}_{-1}$
  \\
  
  $v_{0}$ & $= x_2$
  &
  $\bar{x}_2$ & $= \bar{v}_{0}$
  \\

  \hline

  $v_1$ & $= \ln{v_{-1}}$
  &
  \(\bar{v}_{-1}\)
  &
  \(= \bar{v}_{-1} + \diff{1}{-1}\)
  &
  \(= \bar{v}_{-1} + \bar{v}_1 / v_{-1}
  \) \\

  $v_2$ & $= v_{-1} \times v_0$
  &
  \(\bar{v}_0\)
  &
  \(= \bar{v}_0 + \diff{2}{0}\)
  &
  \(= \bar{v}_0 + \bar{v}_2 \times v_{-1}
  \) \\

  &
  &
  \(\bar{v}_{-1}\)
  &
  \(= \diff{2}{-1}\)
  &
  \(
  = \bar{v}_2 \times v_{0}
  \) \\

  $v_3$ & $= \sin{v_0}$
  &
  \(\bar{v}_0\)
  &
  \(= \diff{3}{0}\)
  &
  \(
  = \bar{v}_3 \times \cos v_0
  \) \\

  $v_4$ & $= v_1 + v_2$
  &
  \(\bar{v}_2\)
  &
  \(= \diff{4}{2}\)
  &
  \(
  = \bar{v}_4 \times 1
  \) \\

  &
  &
  \(\bar{v}_1\)
  &
  \(= \diff{4}{1}\)
  &
  \(
  = \bar{v}_4 \times 1
  \) \\

  $v_5$ & $= v_4 - v_3$
  &
  \(\bar{v}_3\)
  &
  \(= \diff{5}{3}\)
  &
  \(
  = \bar{v}_5 \times (-1)
  \) \\
  
  &
  &
  \(\bar{v}_4\)
  &
  \(= \diff{5}{4}\)
  &
  \(
  = \bar{v}_5 \times 1
  \) \\
  
  \hline

  $y$ & $= v_5$
  &
  $\bar{v}_5$ & $= \bar{y}$
  \\

\end{tabular}
\caption{\label{adml-diff} The reverse mode derivative of program
  (\ref{adml-program})}
\end{figure}

The left-hand column of Figure \ref{adml-diff} contains the statements
of program (\ref{adml-program}) after preparation.  The right-hand
column contains the derivative statements, each of which corresponds
to a statement on the left-hand side.  The reverse mode derivative
program is formed by taking the statements in the left column in order
followed by the statements in the right column in \emph{reverse}
order.  It takes as inputs $x_1$, $x_2$ and $\bar{y}$ and returns as
outputs $y$, $\bar{x}_1$ and $\bar{x}_2$.

\section{Explicit duplication}

To a pure functional programmer the derivative program might cause
some concern.  It seems to rely on the ability to modify the values of
variables after they have been assigned, in this case \(\bar{v}_{-1}\)
and \(\bar{v}_0\).  Does reverse mode AD inherently depend on
mutability?  Perhaps, but as we shall see, we need not give up purity.

Note that each update to a sensitivity variable $\bar{v}$ occurs on a
line corresponding to a use the primal variable $v$ as an argument
(other than the last such use where a direct assignment to $\bar{v}$
occurs). Thus, the recipe above mixes two concerns: that of
differentiating each line of the source program and that of keeping
track of each use site of each variable.  We can separate the concerns
by performing an additional preparation pass which explicitly tracks
reuse of each variable.

Figure \ref{adml-diff-dup} shows the reverse mode program generated
using ``explicit duplication form''. For every reused variable we
insert an explicit ``$\dup$'' call into the program, replacing the
original variable with two fresh variables.  The reverse derivative of
$\dup$ is $+$ so, in the reverse pass, lines corresponding to
duplication perform the accumulation that was handled by mutation in
the earlier version.  The calculations performed in Figures
\ref{adml-diff} and \ref{adml-diff-dup} are exactly the same.  The
difference between them is is that the code of Figure
\ref{adml-diff-dup} is structured in a way which demonstrates that no
mutation need happen.

\begin{figure}[t]
\center
\begin{tabular}[t]{ll|lll}

  $v_{-1}$ & $= x_1$
  &
  $\bar{x}_1$ & $= \bar{v}_{-1}$
  \\
  
  $v_{0}$ & $= x_2$
  &
  $\bar{x}_2$ & $= \bar{v}_{0}$
  \\

  \hline

  \((v_{-1,1}, v_{-1,2})\) & \(= \dup \, v_{-1}\)
  &
  \(\bar{v}_{-1}\) & \(= \bar{v}_{-1,1} + \bar{v}_{-1,2}\)
  \\

  \((v_{0,1}, v_{0,2})\) & \(= \dup \, v_0\)
  &
  \(\bar{v}_{0}\) & \(= \bar{v}_{0,1} + \bar{v}_{0,2}\)
  \\

  $v_1$ & $= \ln{v_{-1,1}}$
  &
  \(\bar{v}_{-1,1}\)
  & \(
  = \diff{1}{-1,1}\)
  & \(
  = \bar{v}_1 / v_{-1,1}
  \) \\

  $v_2$ & $= v_{-1,2} \times v_{0,1}$
  &
  \(\bar{v}_{0,1}\)
  & \(
  = \diff{2}{0,1}\)
  & \(
  = \bar{v}_2 \times v_{-1,2}
  \) \\

  & &
  \(\bar{v}_{-1,2}\)
  &
  \(= \diff{2}{-1,2}\)
  & \(
  = \bar{v}_2 \times v_{0,1}
  \) \\

  $v_3$ & $= \sin{v_{0,2}}$
  &
  \(\bar{v}_{0,2}\)
  & \(
  = \diff{3}{0,2}\)
  & \(
  = \bar{v}_3 \times \cos v_{0,2}
  \) \\

  $v_4$ & $= v_1 + v_2$
  &
  \(\bar{v}_2\)
  & \(
  = \diff{4}{2}\)
  & \(
  = \bar{v}_4 \times 1
  \) \\

  & &
  \(\bar{v}_1\)
  &
  \(= \diff{4}{1}\)
  & \(
  = \bar{v}_4 \times 1
  \) \\

  $v_5$ & $= v_4 - v_3$
  &
  \(\bar{v}_3\)
  & \(
  = \diff{5}{3}\)
  & \(
  = \bar{v}_5 \times (-1)
  \) \\
  
  & &
  \(\bar{v}_4\)
  & \(
  = \diff{5}{4}\)
  & \(
  = \bar{v}_5 \times 1
  \) \\
  
  \hline

  $y$ & $= v_5$
  &
  $\bar{v}_5$ & $= \bar{y}$
  \\

\end{tabular}
\caption{\label{adml-diff-dup} The reverse mode derivative of program
  (\ref{adml-program}) in explicit duplication style}
\end{figure}

As an example of how the explicit duplication transform is applied,
consider the variable $v_0$.  It was used twice in the primal code.
Thus we insert a statement $(v_{0,1}, v_{0,2}) = \mathrm{dup} \, v_0$
which generates two fresh names with the same value as $v_0$.  We also
update the use sites of $v_0$ so that each uses a unique one of these
names.  The same transformation is done for the variable $v_1$.

Now the input program has the nice property that every variable is not
only defined exactly once but also \emph{used} exactly once (except
for the inputs and outputs).  Our AD pass is simpler, because
everything to do with keeping track of use sites is taken care of
implicitly.  Pure functional programmers are happier because it no
longer looks like we are modifying variables after they have been
bound.

\section{Arrays}

In the previous section explicit duplication style was used as a
convenient preprocessing step to make the AD pass simpler.  It also
assuaged some of the fears of a pure functional programmer regarding
mutation.  We will shortly describe how explicit duplication style can
also be turned to the goal of achieving efficient differentiation of
array programs in a pure language but let us first demonstrate what
the difficulty is.

The problem we are trying to solve occurs when we differentiate
programs that contain the array index expression.  If a statement in
our source program is of the form $e = v[i]$ then in a language that
allows array mutation we could emit the reverse instruction
$\bar{v}[i] = \bar{v}[i] + \bar{e}$ which efficiently modifies
$\bar{v}$ in place.  In language that does not support mutation we do
not have this luxury.  Instead we would have to emit something like
$\bar{v} = \bar{v} + \textrm{deltaVec} \, (\textrm{size} \, v) \, i \,
\bar{e}$, meaning that the variable $\bar{v}$ is shadowed (not its
referent modified) and the new $\bar{v}$'s value is the old
$\bar{v}$'s value plus a ``delta'' array of the same size, where
$\textrm{deltaVec} \, n \, i \, e$ is a vector of size $n$ satisfying

\[
(\textrm{deltaVec} \, n \, i \, e)[j]
= \textrm{ if } i == j \textrm{ then } e  \textrm{ else } 0
\]

The mathematical value of the result is the same in each case.  On the
other hand, the computations performed are very different.
Specifically, mutating a array at a single location has constant time
complexity but adding two arrays has time complexity proportional to
their size.  The reverse mode code containing the $\textrm{deltaVec}$
wastes a lot of time adding zeroes to elements of $\bar{v}$.  The
absence of mutability in our language hurts us even in the cases where
the source program itself is pure!  Optimisations might solve the
problem in particular cases but there is no general solution.

% It's a bit bold to claim there is no general solution.

\section{Purity}

It seems that mutation is essential for efficient reverse mode AD of
array programs.  How do we reconcile this with our desire for purity?
It turns out that explicit duplication style can help.  Let's see how
by examining what purity is, and more importantly, why we might want a
language to be pure.

A pure language is one in which every function is pure.  According to
\cite{purity}, for a function to be pure the following conditions must
hold

\begin{itemize}
  \item
    Its return value is the same for the same arguments (no variation
    with local static variables, non-local variables, mutable
    reference arguments or input streams from I/O devices).

  \item
    Its evaluation has no side effects (no mutation of local
    static variables, non-local variables, mutable reference
    arguments or I/O streams).
\end{itemize}

These purity conditions are desirable because the former permits a
very strong form of ``dead code elimination'' and the latter permits a
very strong form of ``common sub-expression elimination''.  Together
they allow freedom over the order in which function calls are
evaluated.  Having these properties at one's disposal permits a large
class of behavior-preserving program transformations.  The
transformations can be used by a compiler author (for the purpose of
optimisation) and a programmer (for the purpose of refactoring).

In a language which allows array mutation to be observed one does not
have the benefit of freely applying the above transformations.  On the
other hand, if mutation happens but cannot be observed in other parts
of the program then we \emph{can} apply the above transformations and
the benefits of purity continue to hold.

This analysis of the benefits of purity gives us the clue we need to
resolve the apparent tension between reverse mode AD and purity.  The
explicit duplication pass described above ensures that every variable
is used only once.  If a variable referring to an array is used only
once then we can safely mutate it knowing that the mutation can never
be observed!\footnote{assuming that no other variable also references
  the same array, a condition that will indeed be satisfied by the
  transformation pass we will describe} This might sound like a rhetorical
sleight of hand, so let us proceed to immediately demonstrate that
this point of view is actually useful.

\newcommand{\indexL}{\mathrm{index}}
\newcommand{\incL}{\mathrm{inc}}

\begin{figure}[t]
\center
\begin{tabular}[t]{ll|ll}
  $v_0$ & $= v[0]$
  &
  $\bar{v}[0]$ & $ = \bar{v}[0] + \bar{v}_0$
  \\
  
  $v_1$ & $ = v[1]$
  &
  $\bar{v}[1]$ & $ = \bar{v}[1] + \bar{v}_1$
  \\

  $r$ & $= v_0 * v_1$
  &
  $\bar{v}_0$ & $ = v_1 * \bar{r}$
  \\

  & &
  $\bar{v}_1$ & $ = v_0 * \bar{r}$
  \\
\end{tabular}
\caption{\label{array-program-mutating} The reverse mode derivative of
  a pure array program, using mutation}
\end{figure}

Consider a program, for example the one in the left-hand column of
Figure \ref{array-program-mutating}, in a pure first-order language
containing the array index operation
\[
\cdot[\cdot] : \mathrm{Array} \, a \to \mathrm{Int} \to a
\]
Its reverse derivative will contain array mutation whenever the primal
program performed array indexing.  Now replace all array index calls
in the primal program with calls to the function
\[
\indexL : \mathrm{Array} \, a \to \mathrm{Int} \to (a, \mathrm{Array}
\, a)
\]
which returns the element of the array at the given index as well as
the original array unchanged.  This allows us to mention the original
array later in the program, via the name given to the second component
of the return tuple, without explicitly duplicating it\footnote{which
  would amount to a copy, for semantic reasons that we won't elaborate
  on in this note}.  Reuses of scalar variables can then be removed by
transforming to explicit duplication style.

\begin{figure}[t]
\center
\begin{tabular}[t]{ll|ll}
  $(v_0, v')$ & $ = \indexL \, v \, 0$
  &
  $\bar{v}$ & $ = \mathrm{inc} \, \bar{v}' \, 0 \, \bar{v}_0$
  \\
  
  $(v_1, v'')$ & $ = \indexL \, v' \, 1$
  &
  $\bar{v}'$ & $ = \mathrm{inc} \, \bar{v}'' \, 1 \, \bar{v}_1$
  \\

  $r$ & $ = v_0 * v_1$
  &
  $\bar{v}_0$ & $ = v_1 * \bar{r}$
  \\

  & &
  $\bar{v}_1$ & $ = v_0 * \bar{r}$
  \\
\end{tabular}
\caption{\label{array-program-dup} The reverse mode derivative of
  a pure array program, in the explicit duplication style using
  unobservable mutation}
\end{figure}

After this transformation the primal program has the property that
every variable is used exactly once, implying that no
\emph{observable} in-place accumulation will be needed in the reverse
program.  The reverse mode derivative of $\indexL$ is $\incL$ which
performs in-place accumulation to an array at a given index, returning the array
after accumulation.
\[
\incL : \mathrm{Array} \, a \to \mathrm{Int} \to a \to \mathrm{Array}
\, a
\]
% or just substitute Float instead of a
Recall that the input array to $\textrm{inc}$
cannot subsequently be used in the program so the mutation that
$\incL$ performs cannot be observed.  Figure \ref{array-program-dup}
shows the program differentiated in terms of $\incL$.
We hypothesise that any pure first-order program with array indexing
can be rewritten to use $\indexL$ instead of $\cdot[\cdot]$.

We have generated two different programs, in Figures
\ref{array-program-mutating} and \ref{array-program-dup}, which
perform exactly the same calculations.  The former involves explicit
mutation that could, in principle, be observed in the rest of the
program.  The latter shows that the mutation can be made unobservable
to the rest of the program, preserving the desirable properties of
purity.

\section{Conclusion and further work}

In this note we presented an approach which, we hypothesise, can
reconcile reverse mode AD (which seems to inherently require mutation)
with a desire for purity (which facilitates a large class of useful
program transformations).  We present only the details needed to
handle straight line programs (no loops or recursion, no branches, no
higher-order functions).  Explicit duplication style is a very weak
form of linear typing and the ideas presented here extend in a natural
way to the general case, taking further inspiration from the theory of
linear types.

%% \section{Leftover bits}

%% Put differently with greater symmetry

%% \begin{itemize}
%%  \item Duplicating the result of a computation is the same as running
%%    it twice

%%  \item Discarding the result of a computation is the same as not
%%    running it at all
%% \end{itemize}

%% The ability to mutate arrays implies that the duplication operation on
%% arrays must perform a copy.  We need to avoid this copy otherwise we
%% are back to square one!

\section{Acknowledgements}

We are grateful to Tom Minka for bringing to our attention the
benefits of explicit duplication style in his talk \cite{minka}.

\begin{thebibliography}{9}

\bibitem{adml}
  Automatic differentiation in machine learning: a survey;
  Atilim Gunes Baydin, Barak A. Pearlmutter, Alexey Andreyevich Radul, Jeffrey Mark Siskind
  
https://arxiv.org/abs/1502.05767
  
\bibitem{purity}
  https://en.wikipedia.org/wiki/Pure\_function

  TODO: Really ought to find a better reference.  Maybe one of the two
  books that are referenced in Wikipedia

\bibitem{minka}
  From automatic differentiation to message passing; Tom Minka

  https://tminka.github.io/papers/acmll2019/

\end{thebibliography}

\end{document}
